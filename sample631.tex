%% Beginning of file 'sample631.tex'
%%
%% Modified 2021 March
%%
%% This is a sample manuscript marked up using the
%% AASTeX v6.31 LaTeX 2e macros.
%%
%% AASTeX is now based on Alexey Vikhlinin's emulateapj.cls 
%% (Copyright 2000-2015).  See the classfile for details.

%% AASTeX requires revtex4-1.cls and other external packages such as
%% latexsym, graphicx, amssymb, longtable, and epsf.  Note that as of 
%% Oct 2020, APS now uses revtex4.2e for its journals but remember that 
%% AASTeX v6+ still uses v4.1. All of these external packages should 
%% already be present in the modern TeX distributions but not always.
%% For example, revtex4.1 seems to be missing in the linux version of
%% TexLive 2020. One should be able to get all packages from www.ctan.org.
%% In particular, revtex v4.1 can be found at 
%% https://www.ctan.org/pkg/revtex4-1.

%% The first piece of markup in an AASTeX v6.x document is the \documentclass
%% command. LaTeX will ignore any data that comes before this command. The 
%% documentclass can take an optional argument to modify the output style.
%% The command below calls the preprint style which will produce a tightly 
%% typeset, one-column, single-spaced document.  It is the default and thus
%% does not need to be explicitly stated.
%%
%% using aastex version 6.3
\documentclass[linenumbers, twocolumn, times]{aastex631}

%% The default is a single spaced, 10 point font, single spaced article.
%% There are 5 other style options available via an optional argument. They
%% can be invoked like this:
%%
%% \documentclass[arguments]{aastex631}
%% 
%% where the layout options are:
%%
%%  twocolumn   : two text columns, 10 point font, single spaced article.
%%                This is the most compact and represent the final published
%%                derived PDF copy of the accepted manuscript from the publisher
%%  manuscript  : one text column, 12 point font, double spaced article.
%%  preprint    : one text column, 12 point font, single spaced article.  
%%  preprint2   : two text columns, 12 point font, single spaced article.
%%  modern      : a stylish, single text column, 12 point font, article with
%% 		  wider left and right margins. This uses the Daniel
%% 		  Foreman-Mackey and David Hogg design.
%%  RNAAS       : Supresses an abstract. Originally for RNAAS manuscripts 
%%                but now that abstracts are required this is obsolete for
%%                AAS Journals. Authors might need it for other reasons. DO NOT
%%                use \begin{abstract} and \end{abstract} with this style.
%%
%% Note that you can submit to the AAS Journals in any of these 6 styles.
%%
%% There are other optional arguments one can invoke to allow other stylistic
%% actions. The available options are:
%%
%%   astrosymb    : Loads Astrosymb font and define \astrocommands. 
%%   tighten      : Makes baselineskip slightly smaller, only works with 
%%                  the twocolumn substyle.
%%   times        : uses times font instead of the default
%%   linenumbers  : turn on lineno package.
%%   trackchanges : required to see the revision mark up and print its output
%%   longauthor   : Do not use the more compressed footnote style (default) for 
%%                  the author/collaboration/affiliations. Instead print all
%%                  affiliation information after each name. Creates a much 
%%                  longer author list but may be desirable for short 
%%                  author papers.
%% twocolappendix : make 2 column appendix.
%%   anonymous    : Do not show the authors, affiliations and acknowledgments 
%%                  for dual anonymous review.
%%
%% these can be used in any combination, e.g.
%%
%% \documentclass[twocolumn,linenumbers,trackchanges]{aastex631}
%%
%% AASTeX v6.* now includes \hyperref support. While we have built in specific
%% defaults into the classfile you can manually override them with the
%% \hypersetup command. For example,
%%
%% \hypersetup{linkcolor=red,citecolor=green,filecolor=cyan,urlcolor=magenta}
%%
%% will change the color of the internal links to red, the links to the
%% bibliography to green, the file links to cyan, and the external links to
%% magenta. Additional information on \hyperref options can be found here:
%% https://www.tug.org/applications/hyperref/manual.html#x1-40003
%%
%% Note that in v6.3 "bookmarks" has been changed to "true" in hyperref
%% to improve the accessibility of the compiled pdf file.
%%
%% If you want to create your own macros, you can do so
%% using \newcommand. Your macros should appear before
%% the \begin{document} command.
%%
\usepackage{xspace}

\newcommand{\vdag}{(v)^\dagger}
\newcommand\aastex{AAS\TeX}
\newcommand\latex{La\TeX}
\newcommand{\CCH}{C$_2$H\xspace}
\newcommand{\CCD}{C$_2$D\xspace}

%% Reintroduced the \received and \accepted commands from AASTeX v5.2
%\received{March 1, 2021}
%\revised{April 1, 2021}
%\accepted{\today}

%% Command to document which AAS Journal the manuscript was submitted to.
%% Adds "Submitted to " the argument.
%\submitjournal{PSJ}

%% For manuscript that include authors in collaborations, AASTeX v6.31
%% builds on the \collaboration command to allow greater freedom to 
%% keep the traditional author+affiliation information but only show
%% subsets. The \collaboration command now must appear AFTER the group
%% of authors in the collaboration and it takes TWO arguments. The last
%% is still the collaboration identifier. The text given in this
%% argument is what will be shown in the manuscript. The first argument
%% is the number of author above the \collaboration command to show with
%% the collaboration text. If there are authors that are not part of any
%% collaboration the \nocollaboration command is used. This command takes
%% one argument which is also the number of authors above to show. A
%% dashed line is shown to indicate no collaboration. This example manuscript
%% shows how these commands work to display specific set of authors 
%% on the front page.
%%
%% For manuscript without any need to use \collaboration the 
%% \AuthorCollaborationLimit command from v6.2 can still be used to 
%% show a subset of authors.
%
%\AuthorCollaborationLimit=2
%
%% will only show Schwarz & Muench on the front page of the manuscript
%% (assuming the \collaboration and \nocollaboration commands are
%% commented out).
%%
%% Note that all of the author will be shown in the published article.
%% This feature is meant to be used prior to acceptance to make the
%% front end of a long author article more manageable. Please do not use
%% this functionality for manuscripts with less than 20 authors. Conversely,
%% please do use this when the number of authors exceeds 40.
%%
%% Use \allauthors at the manuscript end to show the full author list.
%% This command should only be used with \AuthorCollaborationLimit is used.

%% The following command can be used to set the latex table counters.  It
%% is needed in this document because it uses a mix of latex tabular and
%% AASTeX deluxetables.  In general it should not be needed.
%\setcounter{table}{1}

%%%%%%%%%%%%%%%%%%%%%%%%%%%%%%%%%%%%%%%%%%%%%%%%%%%%%%%%%%%%%%%%%%%%%%%%%%%%%%%%
%%
%% The following section outlines numerous optional output that
%% can be displayed in the front matter or as running meta-data.
%%
%% If you wish, you may supply running head information, although
%% this information may be modified by the editorial offices.
\shorttitle{C$_2$D in TW Hya}
\shortauthors{Yamato et al.}
%%
%% You can add a light gray and diagonal water-mark to the first page 
%% with this command:
%% \watermark{text}
%% where "text", e.g. DRAFT, is the text to appear.  If the text is 
%% long you can control the water-mark size with:
%% \setwatermarkfontsize{dimension}
%% where dimension is any recognized LaTeX dimension, e.g. pt, in, etc.
%%
%%%%%%%%%%%%%%%%%%%%%%%%%%%%%%%%%%%%%%%%%%%%%%%%%%%%%%%%%%%%%%%%%%%%%%%%%%%%%%%%
\graphicspath{{./}{figures/}}
%% This is the end of the preamble.  Indicate the beginning of the
%% manuscript itself with \begin{document}.

\begin{document}

\title{Deuterium Fractionation of C$_2$H in the TW Hya Disk}

%% LaTeX will automatically break titles if they run longer than
%% one line. However, you may use \\ to force a line break if
%% you desire. In v6.31 you can include a footnote in the title.

%% A significant change from earlier AASTEX versions is in the structure for 
%% calling author and affiliations. The change was necessary to implement 
%% auto-indexing of affiliations which prior was a manual process that could 
%% easily be tedious in large author manuscripts.
%%
%% The \author command is the same as before except it now takes an optional
%% argument which is the 16 digit ORCID. The syntax is:
%% \author[xxxx-xxxx-xxxx-xxxx]{Author Name}
%%
%% This will hyperlink the author name to the author's ORCID page. Note that
%% during compilation, LaTeX will do some limited checking of the format of
%% the ID to make sure it is valid. If the "orcid-ID.png" image file is 
%% present or in the LaTeX pathway, the OrcID icon will appear next to
%% the authors name.
%%
%% Use \affiliation for affiliation information. The old \affil is now aliased
%% to \affiliation. AASTeX v6.31 will automatically index these in the header.
%% When a duplicate is found its index will be the same as its previous entry.
%%
%% Note that \altaffilmark and \altaffiltext have been removed and thus 
%% can not be used to document secondary affiliations. If they are used latex
%% will issue a specific error message and quit. Please use multiple 
%% \affiliation calls for to document more than one affiliation.
%%
%% The new \altaffiliation can be used to indicate some secondary information
%% such as fellowships. This command produces a non-numeric footnote that is
%% set away from the numeric \affiliation footnotes.  NOTE that if an
%% \altaffiliation command is used it must come BEFORE the \affiliation call,
%% right after the \author command, in order to place the footnotes in
%% the proper location.
%%
%% Use \email to set provide email addresses. Each \email will appear on its
%% own line so you can put multiple email address in one \email call. A new
%% \correspondingauthor command is available in V6.31 to identify the
%% corresponding author of the manuscript. It is the author's responsibility
%% to make sure this name is also in the author list.
%%
%% While authors can be grouped inside the same \author and \affiliation
%% commands it is better to have a single author for each. This allows for
%% one to exploit all the new benefits and should make book-keeping easier.
%%
%% If done correctly the peer review system will be able to
%% automatically put the author and affiliation information from the manuscript
%% and save the corresponding author the trouble of entering it by hand.

%\correspondingauthor{August Muench}
%\email{greg.schwarz@aas.org, gus.muench@aas.org}

\author[0000-0003-4099-6941]{Yoshihide Yamato}
\affiliation{Department of Astronomy, Graduate School of Science, The University of Tokyo, 7-3-1 Hongo, Bunkyo, Tokyo 113-0033, Japan}

%% Note that the \and command from previous versions of AASTeX is now
%% depreciated in this version as it is no longer necessary. AASTeX 
%% automatically takes care of all commas and "and"s between authors names.

%% AASTeX 6.31 has the new \collaboration and \nocollaboration commands to
%% provide the collaboration status of a group of authors. These commands 
%% can be used either before or after the list of corresponding authors. The
%% argument for \collaboration is the collaboration identifier. Authors are
%% encouraged to surround collaboration identifiers with ()s. The 
%% \nocollaboration command takes no argument and exists to indicate that
%% the nearby authors are not part of surrounding collaborations.

%% Mark off the abstract in the ``abstract'' environment. 
\begin{abstract}

\textcolor{red}{abstract here.} 250 word limit

\end{abstract}

%% Keywords should appear after the \end{abstract} command. 
%% The AAS Journals now uses Unified Astronomy Thesaurus concepts:
%% https://astrothesaurus.org
%% You will be asked to selected these concepts during the submission process
%% but this old "keyword" functionality is maintained in case authors want
%% to include these concepts in their preprints.
%% \keywords{Classical Novae (251) --- Ultraviolet astronomy(1736) --- History of astronomy(1868) --- Interdisciplinary astronomy(804)}

%% From the front matter, we move on to the body of the paper.
%% Sections are demarcated by \section and \subsection, respectively.
%% Observe the use of the LaTeX \label
%% command after the \subsection to give a symbolic KEY to the
%% subsection for cross-referencing in a \ref command.
%% You can use LaTeX's \ref and \label commands to keep track of
%% cross-references to sections, equations, tables, and figures.
%% That way, if you change the order of any elements, LaTeX will
%% automatically renumber them.
%%
%% We recommend that authors also use the natbib \citep
%% and \citet commands to identify citations.  The citations are
%% tied to the reference list via symbolic KEYs. The KEY corresponds
%% to the KEY in the \bibitem in the reference list below. 

\section{Introduction} \label{sec:intro}

\section{Observations} \label{sec:observation}
\subsection{\CCD}
We retrieve the \CCD $N=2$--1 data in Band~4 from the ALMA Science Archive (project code: 2016.1.00440.S, PI: R.\ Teague). This project consists of three execution blocks (EBs), which were executed on 2016 October 22, 25, and 27 with 39, 38, and 40 antennas, respectively. The shortest baseline length was 18.6 m for all EBs, while the longest baseline length was 1.4, 1.2, and 1.1 km, respectively. For each EB, the integration time was 47 min. The quasar J1037-2934 was used as a bandpass, phase, and flux calibrator for the first EB. For the second and third EBs, J1037-2934 was used as a bandpass and phase calibrator, while as a flux calibrator the quasar J1107-4449 was used. Six \CCD $N=2$--1 hyperfine transitions around ${\approx}144.24$\,GHz was covered by the fourth spectral window in baseband 2, whose spectral resolution was 122 kHz (or ${\approx}0.25$ km s$^{-1}$). 

As the initial step of data reduction, we ran the script \texttt{scriptForPI.py} to restore the pipeline-calibrated visibilities using Common Astronomy Software Application (CASA; \citealt{CASA}) version 4.7.2. We subsequently performed self-calibration using the continuum emission. We used the CASA version 6.6.3 (modular package) for self-calibration and subsequent imaging. We followed the standard self-calibration procedure tailored to the disk observations \citep[e.g.,][]{Andrews2018, Oberg2021_MAPS, Ohashi2023}. This generally includes the alignment of the disk center and flux rescaling among EBs, but for this particular data we did not apply any flux rescaling given that the visibility amplitudes well align among EBs within the typical flux calibration uncertainty of 10\%. We performed five rounds of phase self-calibration with solution intervals of duration of scans, 200, 100, 50, and 25\,s, and one round of amplitude self-calibration with a solution interval of duration of scans. The solutions were then applied to the spectral line visibilities, and the continuum emission was subtracted by a linear fit to the line-free channels on the visibility plane using the \texttt{uvcontsub} task. 

The self-calibrated \CCD $N=2$--1 spectral window was imaged by the CLEAN algorithm \citep{Hogbom1974} implemented in the \texttt{tclean} task. We adopted a channel width of 0.26 km s$^{-1}$ and used a modified Briggs' weighting scheme of \texttt{briggsbwtaper} with a robust parameter of 0.5 and a visibility tapering that achieve a 0\farcs5 circular beam. This tapering parameter was determined by iteratively minimizing the metric of the beam (see \citealt{Czekala2021}). As a CLEAN mask, we used a Keplerian mask, which is generated using the \texttt{keplerian\_mask.py} script \citep{Teague2020} taking into account the multiple hyperfine transitions of \CCD $N=2$--1. The latest stellar and disk geometry parameters are adopted for the mask based on \citet{Teague2022}: $M_\star = 0.82M_\odot$, $\mathrm{P.A.}=151\fdg6$, $i=5\fdg8$, $d=60.1$\,pc, and $v_\mathrm{sys}=2.84$\,km s$^{-1}$. To fully encompass the emission expected from the distribution of \CCH emission in previous studies \citep{Bergin2016}, an inner radius of 0\farcs25 and an outer radius of 2\farcs5 were adopted for the mask. The root-mean-square (RMS) noise level measured outside this mask was 1.1 mJy beam$^{-1}$. The primary beam correction was finally applied. 

% We also assessed the effect of the non-Gaussianity of the dirty beam, so-called JvM effect \citep[see][]{Jorsater1995, Czekala2021, Casassus2022}. The JvM $\epsilon$ (the solid angle ratio between the CLEAN beam and the dirty beam) of the image was deviated from unity only by $\lesssim10$\%, indicating the effect is minimal. We will consider the uncertainty owing to this effect in deriving the \CCD column density in Section \ref{}.

\subsection{C$_2$H}
We used the ALMA archival data of \CCH $N=4$--3 in Band 7 (project code: 2013.1.00198.S, PI: E. Bergin), originally published in \citet{Bergin2016}. The readers are referred to \citet{Bergin2016} for observational details. Six hyperfine lines of \CCH $N=4$--3 are observed by the first spectral window in baseband 1, whose spectral resolution was 144 kHz (${\approx}0.12$\,km\,s$^{-1}$). First we performed the pipeline calibration by running the script \texttt{scriptForPI.py} using CASA version 4.2.2. We subsequently performed self-calibration using the continuum emission in the same way as for \CCD data. Four rounds of phase self-calibration with solution intervals of duration of scans, 200, 100, and 50\,s were performed, followed by one round of amplitude self-calibration with a solution interval of duration of scans. The solutions were then applied to the spectral line visibilities. After subtracting the continuum emission, the self-calibrated visibilities of \CCH $N=4$--3 line was imaged using \texttt{tclean} task with a channel width of 0.12\,km\,s$^{-1}$ and a Briggs' robust parameter of 0.5 jointed with a visibility tapering, which achieved a 0\farcs5 circularized beam matched with the beam size of \CCD data. We used the Keplerian mask generated with the same parameters as for \CCD as a CLEAN mask. The RMS noise level measured outside this mask was 5.0 mJy beam$^{-1}$. We finally applied the primary beam correction.

For both \CCD and \CCH data, the absolute flux calibration accuracy is ${\sim}$10--15\%. The following analyses consider only the statistical uncertainty and does not include the uncertainty owing to this absolute flux calibration accuracy.


\section{Analysis \& results} \label{sec:analysis}

\subsection{Detection of \CCD}
Figure \ref{fig:TW_Hya_gallery} summarizes the observational results of \CCH and \CCD in TW~Hya. We integrated the emission of \CCH $N=4$--3 and \CCD $N=2$--1 along the spectral axis to create the velocity-integrated intensity (or zeroth moment) maps, which are shown in Figure \ref{fig:TW_Hya_gallery} (a) and (b). The Keplerian mask for CLEAN is used and all hyperfine components are stacked. The \CCH emission clearly shows a ring-like morphology with an inner radius of ${\sim}0\farcs5$ and an outer radius of ${\sim}2\arcsec$, which is consistent with the previous studies \citep{Bergin2016, Bergin2024}. The \CCD zeroth moment map exhibit a weak emission, whose spatial distribution is roughly consistent with that of \CCH. We create azimuthally averaged radial profiles of these zeroth moment maps using the Python package \texttt{GoFish} \citep{GoFish}. The \CCH radial profile peaks at ${\approx}70$\,au, where the \CCD also shows a weak but significant emission. This radial coincidence indicates that the emission of \CCH and \CCD trace the similar radial region. We also show the deprojected, azimuthally stacked spectra of three detected \CCD $N=2$--1 hyperfine transitions in Figure \ref{fig:TW_Hya_gallery} (d). Prior to the stacking, we align the velocity shift due to the Keplerian rotation to the systemic velocity (2.8\,km\,s$^{-1}$; \citealt{Teague2022}) using \texttt{GoFish} \citep{GoFish} assuming the same stellar and disk parameters as used for Keplerian mask. The spectrum clearly exhibits  detection of the $F=3/2$--$1/2$ and $F=5/2$--$3/2$ hyperfine component, which are blended, at a significance of ${\gtrsim}5\sigma$, while the $F=7/2$--$5/2$ hyperfine component is at ${\sim}3\sigma$ significance. Other three hyperfine components are not detected due to their weaker line strengths (i.e., small Einstein A coefficients). 

\begin{figure*}
\epsscale{1.1}
\plotone{TW_Hya_gallery.pdf}
\caption{Velocity-integrated intensity (zeroth moment) map of \CCH $N=4$--3 (a) and \CCD $N=2$--1 (b) stacked over all hyperfine lines. In each panel, the 0\farcs5 synthesized beam and a 50 au scale bar are shown in the lower left and lower right, respectively. The panel (c) shows the azimuthally averaged radial profiles of the \CCH and \CCD emission. The shaded regions indicate uncertainties of [1, 2, 3]$\sigma$. The panel (d) shows the deprojected, azimuthally stacked spectrum of \CCD $N=2$--1 over an annulus with an inner radius of 0\farcs8 and an outer radius of 1\farcs8. While the observations cover six hyperfine lines of \CCD, here only the detected ones among them ($F=3/2$--$1/2$, $F=5/2$--$3/2$, and $F=7/2$--$5/2$, where the first two are blended) are shown. The LSRK velocity is with respect to the rest frequency of the $F=7/2$--$5/2$ transition, where the systemic velocity of TW~Hya is 2.8\,km\,s$^{-1}$ \citep{Teague2022}.}
\label{fig:TW_Hya_gallery}
\end{figure*}

We also verify the detection of \CCD $N=2$--1 by matched filtering analysis \citep{Loomis2018}. The Keplerian mask for CLEAN is used as a filter kernel, where the inner and outer radii are fixed to 1\farcs0 and 1\farcs6, respectively, to better match to the distribution of the \CCD emission. The response to this filter is computed using the Python package \texttt{VISIBLE} \citep{Loomis2018}, and shown in Figure \ref{fig:matched_filter_C2D}. The blended $F=3/2$--$1/2$ and $F=5/2$--$3/2$ hyperfine components exhibit a ${\gtrsim}6\sigma$ response, while the $F=7/2$--$5/2$ component shows a ${\sim}3\sigma$ response. Other three hyperfine components show no more than $3\sigma$ response as expected from their weaker intrinsic line strength as shown in Figure \ref{fig:matched_filter_C2D}. 


\begin{figure}
\plotone{TW_Hya_response_rmin1.0_rmax1.6.pdf}
\caption{Matched filter responses of \CCD $N=2$--1 hyperfine transitions. The filter kernel is a simple Keplerian-rotating disk model with the inner and outer radius of 1\farcs0 and 1\farcs6, respectively. The gray dotted horizontal line indicate the $3\sigma$ response. The frequency and intrinsic line strength for each hyperfine component is shown by the vertical line segments.}
\label{fig:matched_filter_C2D}
\end{figure}


\subsection{Non-LTE Spectral Modeling of \CCH and \CCD}

\section{Discussion} \label{sec:discussion}

\subsection{Deuteration Pathways Traced by \CCH/\CCD Ratio}

\subsection{Multiple Paths of Deuterium Fractionation in the TW~Hya Disk}

\subsection{Effect of \CCH $N=4$--3 Optical Depth on the Published $^{13}$CCH/$^{12}$CCH Ratio}

\section{Summary} \label{sec:summary}


\begin{acknowledgments}
\textcolor{red}{We thank...}
\end{acknowledgments}

%% To help institutions obtain information on the effectiveness of their 
%% telescopes the AAS Journals has created a group of keywords for telescope 
%% facilities.
%
%% Following the acknowledgments section, use the following syntax and the
%% \facility{} or \facilities{} macros to list the keywords of facilities used 
%% in the research for the paper.  Each keyword is check against the master 
%% list during copy editing.  Individual instruments can be provided in 
%% parentheses, after the keyword, but they are not verified.

\vspace{5mm}
\facilities{\textcolor{red}{facilities used here}}

%% Similar to \facility{}, there is the optional \software command to allow 
%% authors a place to specify which programs were used during the creation of 
%% the manuscript. Authors should list each code and include either a
%% citation or url to the code inside ()s when available.

\software{\textcolor{red}{software references here}}

%% Appendix material should be preceded with a single \appendix command.
%% There should be a \section command for each appendix. Mark appendix
%% subsections with the same markup you use in the main body of the paper.

%% Each Appendix (indicated with \section) will be lettered A, B, C, etc.
%% The equation counter will reset when it encounters the \appendix
%% command and will number appendix equations (A1), (A2), etc. The
%% Figure and Table counter will not reset.

\appendix





%% For this sample we use BibTeX plus aasjournals.bst to generate the
%% the bibliography. The sample631.bib file was populated from ADS. To
%% get the citations to show in the compiled file do the following:
%%
%% pdflatex sample631.tex
%% bibtext sample631
%% pdflatex sample631.tex
%% pdflatex sample631.tex

\bibliography{sample631}{}
\bibliographystyle{aasjournal}

%% This command is needed to show the entire author+affiliation list when
%% the collaboration and author truncation commands are used.  It has to
%% go at the end of the manuscript.
%\allauthors

%% Include this line if you are using the \added, \replaced, \deleted
%% commands to see a summary list of all changes at the end of the article.
%\listofchanges

\end{document}

% End of file `sample631.tex'.
